  \documentclass[%
oneside,                 % oneside: electronic viewing, twoside: printing
final,                   % draft: marks overfull hboxes, figures with paths
10pt]{article}

\listfiles               %  print all files needed to compile this document

\usepackage[a4paper, total={6in, 8in}]{geometry}
\usepackage[totoc]{idxlayout}   % for index in the toc
\usepackage[nottoc]{tocbibind}  % for references/bibliography in the toc

\usepackage{relsize,makeidx,color,setspace,amsmath,amsfonts,amssymb}
\usepackage[table]{xcolor}
\usepackage{bm,ltablex,microtype}
\usepackage{comment} 
\usepackage[pdftex]{graphicx}

\usepackage{fancyvrb} % packages needed for verbatim environments

\usepackage[T1]{fontenc}
%\usepackage[latin1]{inputenc}
\usepackage{ucs}
\usepackage[utf8x]{inputenc}

\usepackage{lmodern}         % Latin Modern fonts derived from Computer Modern


\usepackage{pgfplotstable, booktabs}

\pgfplotstableset{
    every head row/.style={before row=\toprule,after row=\midrule},
    every last row/.style={after row=\bottomrule}
}





% Hyperlinks in PDF:
\definecolor{linkcolor}{rgb}{0,0,0.4}
\usepackage{hyperref}
\hypersetup{
    breaklinks=true,
    colorlinks=true,
    linkcolor=linkcolor,
    urlcolor=linkcolor,
    citecolor=black,
    filecolor=black,
    %filecolor=blue,
    pdfmenubar=true,
    pdftoolbar=true,
    bookmarksdepth=3   % Uncomment (and tweak) for PDF bookmarks with more levels than the TOC
    }
%\hyperbaseurl{}   % hyperlinks are relative to this root

\setcounter{tocdepth}{2}  % levels in table of contents

% --- fancyhdr package for fancy headers ---
\usepackage{fancyhdr}
\fancyhf{} % sets both header and footer to nothing
\renewcommand{\headrulewidth}{0pt}
\fancyfoot[LE,RO]{\thepage}
% Ensure copyright on titlepage (article style) and chapter pages (book style)
\fancypagestyle{plain}{
  \fancyhf{}
  \fancyfoot[C]{{\footnotesize \copyright\ 1999-2018, "Computational Physics I FYS3150/FYS4150":"http://www.uio.no/studier/emner/matnat/fys/FYS3150/index-eng.html". Released under CC Attribution-NonCommercial 4.0 license}}
%  \renewcommand{\footrulewidth}{0mm}
  \renewcommand{\headrulewidth}{0mm}
}
% Ensure copyright on titlepages with \thispagestyle{empty}
\fancypagestyle{empty}{
  \fancyhf{}
  \fancyfoot[C]{{ }}
  \renewcommand{\footrulewidth}{0mm}
  \renewcommand{\headrulewidth}{0mm}
}

\pagestyle{fancy}


% prevent orhpans and widows
\clubpenalty = 10000
\widowpenalty = 10000

% --- end of standard preamble for documents ---


% insert custom LaTeX commands...

\raggedbottom
\makeindex
\usepackage[totoc]{idxlayout}   % for index in the toc
\usepackage[nottoc]{tocbibind}  % for references/bibliography in the toc
\usepackage{listings}
\usepackage[normalem]{ulem} 	%for tables
\useunder{\uline}{\ul}{}
\usepackage{hyperref}
\usepackage[section]{placeins} %force figs in section

\usepackage{natbib}

\usepackage[toc,page]{appendix} % appenix
\usepackage{amsmath} % split in align

%-------------------- end preamble ----------------------

\begin{document}

% matching end for #ifdef PREAMBLE

\newcommand{\exercisesection}[1]{\subsection*{#1}}


% ------------------- main content ----------------------



% ----------------- title -------------------------

\thispagestyle{empty}

\begin{center}
{\LARGE\bf
\begin{spacing}{1.25}
VMC%
\end{spacing}
}
\end{center}

% ----------------- author(s) -------------------------

\begin{center}
{\bf Johan Nereng}
\end{center}

    \begin{center}
% List of all institutions:
\centerline{{\small Department of Physics, University of Oslo, Norway}}
\end{center}
    
% ----------------- end author(s) -------------------------

% --- begin date ---
\begin{center}
March 23, 2020
\end{center}
% --- end date ---

\vspace{3cm}
\vspace{3cm}
\begin{abstract}


\end{abstract}


\newpage


\textit{\textbf{Author's comments:} .}
\newpage

\section{Introduction}


In order to write this project paper and the code required to produce the results, I used a variety of tools, including: Python 3.7.5, NumPy \cite{numpy}, , as well as a number of books, web-pages and articles - of which most are listed under 
 \hyperref[refer]{references}. All the code required to reproduce the results may be found on my \href{https://github.com/johanere/FYS-STK4155/tree/master/Project3}{github page }.  



\section{Material and methods} \label{theory}
\subsection{System}
Hard sphere Bose gas for various number of particles.  In order to study the properties of the system, a trial wave function \eqref{eq:trialwf} is used. Trap is an harmonic spherical (S) ($\omega_{ho}^2=\omega_z^2$) or elliptical (E) ($\omega_{ho}^2 \neq \omega_z^2$) one, two and finally three dimensional trap, made up of the potential \eqref{trap_eqn}. The two-body Hamiltionian of the system

\begin{equation}
 V_{ext}(\mathbf{r}) = 
 \Bigg\{
 \begin{array}{ll}
	 \frac{1}{2}m\omega_{ho}^2r^2 & (S)\\
 \strut
	 \frac{1}{2}m[\omega_{ho}^2(x^2+y^2) + \omega_z^2z^2] & (E)
 \label{trap_eqn}
 \end{array}
 \end{equation}
Where $\omega_{ho}^2$ is the trap potential strength.



\begin{equation}
     H = \sum_i^N \left(\frac{-\hbar^2}{2m}{\bigtriangledown }_{i}^2 +V_{ext}({\mathbf{r}}_i)\right)  +
	 \sum_{i<j}^{N} V_{int}({\mathbf{r}}_i,{\mathbf{r}}_j),
	 \label{eq:hamiltonian}
 \end{equation}
 as the two-body Hamiltonian of the system.  Here $\omega_{ho}^2$
 defines the trap potential strength.  In the case of the elliptical
 trap, $V_{ext}(x,y,z)$, $\omega_{ho}=\omega_{\perp}$ is the trap
 frequency in the perpendicular or $xy$ plane and $\omega_z$ the
 frequency in the $z$ direction.  The mean square vibrational
 amplitude of a single boson at $T=0K$ in the trap (\ref{trap_eqn}) is
 $\langle x^2\rangle=(\hbar/2m\omega_{ho})$ so that $a_{ho} \equiv
 (\hbar/m\omega_{ho})^{\frac{1}{2}}$ defines the characteristic length
 of the trap.  The ratio of the frequencies is denoted
 $\lambda=\omega_z/\omega_{\perp}$ leading to a ratio of the trap
 lengths $(a_{\perp}/a_z)=(\omega_z/\omega_{\perp})^{\frac{1}{2}} =
 \sqrt{\lambda}$. Note that we use the shorthand notation
\begin{align}
    \sum_{i < j}^{N} V_{ij} \equiv \sum_{i = 1}^{N}\sum_{j = i + 1}^{N} V_{ij},
\end{align}
 that is, the notation $i < j$ under the summation sign signifies a double sum
 running over all pairwise interactions once.


\begin{equation}
 V_{int}(|\mathbf{r}_i-\mathbf{r}_j|) =  \Bigg\{
 \begin{array}{ll}
	 \infty & {|\mathbf{r}_i-\mathbf{r}_j|} \leq {a}\\
	 0 & {|\mathbf{r}_i-\mathbf{r}_j|} > {a}
 \end{array}
 \end{equation}
 
\begin{equation}
 \Psi_T(\mathbf{r})=\Psi_T(\mathbf{r}_1, \mathbf{r}_2, \dots \mathbf{r}_N,\alpha,\beta)
 =\left[
    \prod_i g(\alpha,\beta,\mathbf{r}_i)
 \right]
 \left[
    \prod_{j<k}f(a,|\mathbf{r}_j-\mathbf{r}_k|)
 \right],
 \label{eq:trialwf}
\end{equation}
Where

\begin{equation}
    g(\alpha,\beta,\mathbf{r}_i)= \exp{[-\alpha(x_i^2+y_i^2+\beta z_i^2)]}.
 \end{equation}
 For spherical traps we have $\beta = 1$ and for non-interacting
 bosons ($a=0$) we have $\alpha = 1/2a_{ho}^2$.  The correlation wave
 function is

\begin{equation}
    f(a,|\mathbf{r}_i-\mathbf{r}_j|)=\Bigg\{
 \begin{array}{ll}
	 0 & {|\mathbf{r}_i-\mathbf{r}_j|} \leq {a}\\
	 (1-\frac{a}{|\mathbf{r}_i-\mathbf{r}_j|}) & {|\mathbf{r}_i-\mathbf{r}_j|} > {a}.
 \end{array}
 \end{equation}

\subsection{Local energy}
The local energy is defined as \eqref{eq:locale}. Using the trial wave function \eqref{eq:trialwf}, the first thingything is to find the analytical expression IOT save flops (quote with ratio).
\begin{equation}
    E_L(\mathbf{r})=\frac{1}{\Psi_T(\mathbf{r})}H\Psi_T(\mathbf{r}),
    \label{eq:locale}
 \end{equation}
 
\subsubsection{Non-interacting}
In the Non-interacting case, the local energy \eqref{eq:locale} on the trial wave functio eqref{eq:trialwf} with $a=0$ -> only HO potential and $\beta=1$. This also means that the internal potential $V_{int}$ in the Hamiltonian \eqref{eq:hamiltonian} becomes zero. 
\begin{equation}
    E_L(\mathbf{r}) \Psi_T(\mathbf{r}) =\frac{1}{ 
    \prod_i g(\alpha,\beta,\mathbf{r}_i)
}  \sum_i^N \left(\frac{-\hbar^2}{2m}{\nabla }_{i}^2 +V_{ext}({\mathbf{r}}_i)\right)  \left[
    \prod_i g(\alpha,\beta,\mathbf{r}_i)
 \right]
    \label{eq:locale}
 \end{equation}
Taking the gradient of $g$:
\begin{equation}
{\nabla }_{i} g(\alpha,\beta,\mathbf{r}_i)=-2\alpha \mathbf{r}_i g(\alpha,\beta,\mathbf{r}_i)
\end{equation}
Taking the Laplacian of the $g$ thing which I'm sure has another name. From the product rule, this entails deriving $ \mathbf{r}_i$, resulting in a coefficient $d$ representing the dimensionality of $r$. 
\begin{equation}
{\nabla }_{i}^2  g(\alpha,\beta,\mathbf{r}_i)=  (-2d \alpha  + 4 \alpha  \mathbf{r}_i^2) g(\alpha,\beta,\mathbf{r}_i)
\label{eq:grad_g}
\end{equation}
Meaning that
\begin{equation}
        E_L(\mathbf{r}) =  \sum_i^N \left(\frac{-\hbar^2}{2m}(-2 d \alpha + 4 \alpha  \mathbf{r}_i^2) + \frac{1}{2}m\omega_{ho}^2r_i^2 )\right)  
    \label{eq:locale}
 \end{equation}
Using natural units, $\hbar=c=1$, and unity mass $m=1$, the equation becomes;

\begin{equation}
        E_L(\mathbf{r}) =  \alpha d N + \left( - 2 \alpha    + \frac{1}{2} \omega_{ho}^2\right)  \sum_i^N r_i^2 
\label{eq:locale_unity}
\end{equation}
 
\eqref{eq:grad_g} also leads to an analytic expression for the drift force used in the importance sampling;
\begin{equation}
F_i = \frac{2\nabla \Psi_T}{\Psi_T}= -4\alpha \mathbf{r}_i 
\end{equation}
 
\subsubsection{Interacting - IKKE FERDIG}
Next, we will find the local energy for the full problem in three dimensions.
The tricky part is to find an analytic expressions for the derivative of the trial wave function

\begin{equation*}
   \frac{1}{\Psi_T(\mathbf{r})}\sum_i^{N}\nabla_i^2\Psi_T(\mathbf{r}),
\end{equation*}
with the above 
trial wave function of Eq. (\ref{eq:trialwf}).
We rewrite 

\begin{equation*}
\Psi_T(\mathbf{r})=\Psi_T(\mathbf{r}_1, \mathbf{r}_2, \dots \mathbf{r}_N,\alpha,\beta)
=\left[
    \prod_i g(\alpha,\beta,\mathbf{r}_i)
\right]
\left[
    \prod_{j<k}f(a,|\mathbf{r}_j-\mathbf{r}_k|)
\right],
\end{equation*}
as

\begin{equation*}
\Psi_T(\mathbf{r})=\left[
    \prod_i g(\alpha,\beta,\mathbf{r}_i)
\right]
\exp{\left(\sum_{j<k}u_{jk}\right)}
\end{equation*}
where we have defined $r_{ij}=|\mathbf{r}_i-\mathbf{r}_j|$
and

\begin{equation*}
   \prod_{i<j} f(r_{ij})= \exp{\left(\sum_{i<j}u_{jk}\right)},
\end{equation*}
with $u_{jk}=\ln{f(r_{ij})}$.
We have also
\begin{equation*}
    g(\alpha,\beta,\mathbf{r}_i) = \exp{\left[-\alpha(x_i^2+y_i^2+\beta
    z_i^2)\right]}= \phi(\mathbf{r}_i) = \phi_i.
\end{equation*}

Taking the gradient with respect to the k'th particle;

\begin{align}
  \nabla_k\Psi_T(\mathbf{r}) &= \nabla_k \left(\left[
    \prod_i \phi_i
\right]
\exp{\left(\sum_{j<m}u_{jm}\right)}\right) \\
 &= \left( \nabla_k \left[
    \prod_i \phi_i
\right] \right)
\exp{\left(\sum_{j<m} u_{jm}\right)} +
\left[
    \prod_i \phi_i
\right]
\left( \nabla_k \exp{\left(\sum_{j<m} u_{jm} \right)}\right) \label{AA_gradfullTW}
\end{align}
The gradient of the non-interacting part of the TW:
\begin{align}
& \nabla_k \left[
    \prod_i \phi_i
\right] =  \nabla_k \phi_k \left[
    \prod_{i\neq k} \phi_i
\right]= \nabla_k \phi_k \frac{\prod_i g(\alpha,\beta,\mathbf{r}_i)}{\phi_k} \\
\end{align}
And the gradient of the interacting part, remembering that $r_{kl}=r_{lk}$;
\begin{align}
\nabla_k \exp{\left(\sum_{j<m} u_{jm} \right)}
= \exp{\left(\sum_{j<m} u_{jm} \right)} \sum_{j\neq k} \nabla_k u_{kj}
=  \prod_{j<m} f(r_{jm}) \sum_{l\neq k} \nabla_k u_{kl}
\end{align}

Thus \eqref{AA_gradfullTW} is
\begin{align}
\begin{split}
  \nabla_k\Psi_T(\mathbf{r}) &= \nabla_k\phi_k\left[\prod_{i\ne k}\phi_i\right]\exp{\left(\sum_{j<m}u_{jm}\right)}
  \\
  &\qquad
  +  \left[\prod_i\phi_i\right]
  \exp{\left(\sum_{j<m}u_{jm}\right)}\sum_{l\ne k}\nabla_k u_{kl}
\end{split}
\end{align}
or

\begin{align}
\begin{split}
  \nabla_k\Psi_T(\mathbf{r}) &=\nabla_k \phi_k \frac{\prod_i g(\alpha,\beta,\mathbf{r}_i)}{\phi_k}\prod_{j<m} f(r_{jm})
  \\
  &\qquad
  +  \prod_i  g(\alpha,\beta,\mathbf{r}_i)
 \prod_{j<m} f(r_{jm})\sum_{l\ne k}\nabla_k u_{kl} \\
&= \left(\ \frac{\nabla_k \phi_k}{\phi_k} + \sum_{l\ne k}\nabla_k u_{kl} \right) \Psi_T(\mathbf{r})  
\end{split}
\end{align}
Next, we find the second derivative;

\begin{align}  
\begin{split}
\frac{1}{\Psi_T(\mathbf{r})} \nabla_k^2\Psi_T(\mathbf{r}) 
&=
\frac{1}{\Psi_T(\mathbf{r})} \nabla_k\left( \left(\frac{\nabla_k \phi_k}{\phi_k} + \sum_{l\ne k}\nabla_k u_{kl} \right) \Psi_T(\mathbf{r}) \right)\\
&=
\frac{1}{\Psi_T(\mathbf{r})} \left(\left(\phi_k \nabla_k\frac{1}{\phi_k} +\frac{\nabla_k^2 \phi_k}{\phi_k} + \sum_{l\ne k}\nabla_k^2 u_{kl} \right)\Psi_T(\mathbf{r}) + 
\left(\ \frac{\nabla_k \phi_k}{\phi_k} + \sum_{l\ne k}\nabla_k u_{kl} \right)^2 \Psi_T(\mathbf{r})   \right) \\
&=
 \left(\ \frac{\nabla_k \phi_k}{\phi_k} \right)^2 + \frac{\nabla_k^2 \phi_k}{\phi_k} + \sum_{l\ne k}\nabla_k^2 u_{kl}  + 
\left(\ \frac{\nabla_k \phi_k}{\phi_k} + \sum_{l\ne k}\nabla_k u_{kl} \right)^2 \\
&=
 -\left(\frac{\nabla_k \phi_k}{\phi_k}\right)^2 + \frac{\nabla_k^2 \phi_k}{\phi_k} + \sum_{l\ne k}\nabla_k^2 u_{kl}  + 
\left(\ \frac{\nabla_k \phi_k}{\phi_k} \right)^2 + 2\left(\ \frac{\nabla_k \phi_k}{\phi_k} \sum_{l\ne k}\nabla_k u_{kl} \right)+ \left(\ \sum_{l\ne k}\nabla_k u_{kl} \right)^2 \\
&= 
\frac{\nabla_k^2 \phi_k}{\phi_k} + 2  \frac{\nabla_k \phi_k}{\phi_k} \sum_{l\ne k}\nabla_k u_{kl} + \sum_{l\ne k}\nabla_k^2 u_{kl}   + \left(\ \sum_{l\ne k}\nabla_k u_{kl} \right)^2
\end{split}      
\end{align}
In order to simplify applying the $\nabla_k$-operator to $u_{kl}$, the operator is re-written:
\begin{align*}
\nabla_k = \nabla_k \frac{\partial r_{kl}}{\partial r_{kl}} = \nabla_k \sqrt{\left(\bm r_k - \bm r_l \right)^2} \frac{\partial}{\partial r_{kl}} = \frac{\bm r_k - \bm r_l}{r_{kl}} \frac{\partial}{\partial r_{kl}}
\end{align*}
This re-written operator is then applied to the $\nabla_k u_{kl}$ terms, such that
\begin{align*}
\nabla_k u_{kl} = \frac{\bm r_k - \bm r_l}{r_{kl}} \frac{\partial u_{kl}}{\partial r_{kl}} = \frac{\bm r_k - \bm r_l}{r_{kl}}  u'_{kl}
\end{align*}
And
\begin{align*}
\nabla_k^2 u_{kl} 
&= \left( \nabla_k  \frac{\bm r_k - \bm r_l}{r_{kl}} \right) \partial u'_{kl}  +  \frac{\bm r_k - \bm r_l}{r_{kl}} \left( \nabla_k   u'_{kl}  \right) \\
&=  \left( \frac{ \nabla_k (\bm r_k - \bm r_l)}{r_{kl}} \right)  u'_{kl}  +   (\bm r_k - \bm r_l) \left( \nabla_k \frac{1 }{r_{kl}} \right)  u'_{kl}  +  \frac{\bm r_k - \bm r_l}{r_{kl}} \left( \nabla_k   u'_{kl}  \right) \\
&= 
\frac{d}{r_{kl}}   u'_{kl}  -  (\bm r_k - \bm r_l) \frac{(\bm r_k - \bm r_l)}{r_{kl}^3}   u'_{kl} 
+  \left( \frac{\bm r_k - \bm r_l}{r_{kl}} \right)^2   u{''}_{kl} \\
&= 
\left( \frac{d}{r_{kl}}   -   \frac{(\bm r_k - \bm r_l)^2}{r_{kl}^3} \right)  u'_{kl} 
+  \left( \frac{\bm r_k - \bm r_l}{r_{kl}} \right)^2   u{''}_{kl}
\end{align*}
Where $(\bm r_k - \bm r_l)^2={r_{kl}}^2$, thus
\begin{align*}
\nabla_k^2 u_{kl} 
&=
\left( \frac{d}{r_{kl}}   -   \frac{1}{r_{kl}} \right)  u'_{kl} 
+    u{''}_{kl} =   \frac{d-1}{r_{kl}}   u'_{kl} 
+    u{''}_{kl}
\end{align*}
Applied to the Laplacian;
\begin{align}
\begin{split}
\frac{1}{\Psi_T(\mathbf{r})} \nabla_k^2\Psi_T(\mathbf{r}) 
&=
\frac{\nabla_k^2 \phi_k}{\phi_k} + 2  \frac{\nabla_k \phi_k}{\phi_k} \sum_{l\ne k}\frac{\bm r_k - \bm r_l}{r_{kl}}  u'_{kl}  \\
&+  \left(\ \sum_{l\ne k} \frac{\bm r_k - \bm r_l}{r_{kl}} \partial u'_{kl} \right)^2 \\
&+ \sum_{l\ne k} \left( \frac{d-1}{r_{kl}}   u'_{kl} +   u{''}_{kl} \right)
\end{split}
\end{align}
Expanding the third term, re-arranging, and inserting $d=3$;
\begin{align}
\begin{split}
\frac{1}{\Psi_T(\mathbf{r})} \nabla_k^2\Psi_T(\mathbf{r}) 
&=
\frac{\nabla_k^2 \phi_k}{\phi_k} + 2  \frac{\nabla_k \phi_k}{\phi_k} \sum_{l\ne k}\frac{\bm r_k - \bm r_l}{r_{kl}}  u'_{kl}  \\
&+  \sum_{j\ne k} \sum_{l\ne k} \frac{(\bm r_k - \bm r_l)(\bm r_k - \bm r_j)}{r_{kj} r_{kl} }  u'_{kj}  u'_{kl} \\
&+ \sum_{l\ne k} \left(   u{''}_{kl} + \frac{2}{r_{kl}}   u'_{kl} \right)
\end{split}
\end{align}

 
\section{Results} \label{results}

\section{Conclusions} \label{conclusions}



\bibliography{ref} \label{refer}
\bibliographystyle{plain}


\begin{appendices}
\section*{Appendix 1.} \label{APP_1}


\end{appendices}
\end{document}